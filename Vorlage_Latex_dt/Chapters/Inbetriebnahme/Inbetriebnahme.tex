\chapter{Inbetriebnahme}
\label{cha:Inbetriebnahme}

Die Entwicklung der Regelsoftware und die Inbetriebnahme der Kälteanlage gestaltet sich als ein kleinschrittiger, iterativer Prozess. Abbild \ref{fig:Inbetriebnahmeprozess} zeigt den Inbetriebnahmeprozess. 

\begin{figure}[htb]
\centering	
\includegraphics[width=0.850\textwidth]{Pictures/Inbetriebnahme/Ablauf.pdf}
\caption{Inbetriebnahmeprozess mit Fehlersuche und -behebung}
\label{fig:Inbetriebnahmeprozess}
\end{figure}

Durch die Erweiterung der Kälteanlage durch einen Abtauverdampfer, müssen  auch die elektrischen Schaltschränke verlegt werden. Der Abtauverdampfer wird mittels einer Konstruktion genau über den Verflüssigungssatz montiert. Weitere Rückschlagventile und Sensorstellen werden im Rohrsystem vorgesehen. Zusätzlich wird die Versorgungsmimik für den Verdampfer um zwei weitere Magnetventile ergänzt. Nachdem die Kältefachfirma alle hydraulischen Arbeiten an der Kälteanlage abgeschlossen hat, werden die Rohrleitungen der Kälteanlage durch eine Vakuumpumpe evakuiert und auf Dichtheit geprüft. 

\section{Elektrische Inbetriebnahme}
\label{Inbetriebnahme_ACDC}

Aufgrund der Verlegung der Schaltschränke und Überarbeitung der Funktionen der Kälteanlage, müssen große Teile der Steuerungsschaltschränke erneuert und neue Kabel verlegt werden. Die elektrische Inbetriebnahme der elektrischen Steuerung der Kältemaschine erfolgt durch die Kältefachfirma noch im Manuellen Betrieb ohne die SPS. Hierbei stellt sich heraus, dass das Funktionsmodul des EFC-Frequenzumrichter defekt ist und dadurch der Bock-Kompressor keine Freigabe bekommt. Die Kältefachfirma umgeht diesen Defekt durch eine Spannungsfreigabe auf den Frequenzumrichter mit einem 26-V-Signal aus dem Schaltschrank. Mitgelieferte Sicherheitsfunktionen des Bock-Kompressors werden so außer Kraft gesetzt und müssen deshalb softwaretechnisch im Anlagenschutz implementiert werden. 
Die elektrische Seite der Kälteanlage wird in Betrieb genommen und die Steuerung kann nun Schritt für Schritt auf die SPS umgelegt werden.

Die Spannungsleitung für den Massenstromsensor wird von der Elektrowerkstatt verlegt und in Betrieb genommen. Zusätzlich werden die RS485-Netzwerke mit Knotenpunkte, Abschlusswiderstände, \textit{Pull-up} and \textit{Pull-Down} installiert. Alle Stecker für die Drucksensoren müssen, entsprechend des Handbuches, mit den Stichleitungen verlötet werden. Die Stichleitung jedes Sensors werden dann in einem Knotenpunkt an das Netzwerk angeschlossen. Für zukünftige Modbus RTU-Netzwerke werden RJ45-Stecker empfohlen. Diese beschleunigen die Inbetriebnahme und Auswechselung von Sensoren. Des Weiteren werden Kurzschlüsse ausgeschlossen, da es keine offenen Kabelenden gibt. Ein falscher Anschluss der Datenleitung an die Spannungsversorgung der Drucksensoren kann zur Zerstörung des Sensors führen. Spannungsfreies Arbeiten ist aus diesem Grund unbedingt einzuhalten. 
Die Sensoren können jetzt nacheinander in das RS485-Netzwerk integriert werden. 

\section{Informationstechnische Inbetriebnahme}
\label{sec:Inbetriebnahme_IT}

Zunächst werden alle PT100-Temperatursensoren an die SPS angeschlossen und ausgelesen. 

Danach wird das erste Modbus RTU-Netzwerk in Betrieb genommen. Wichtig hierfür ist, dass alle Sensoren in einem Netzwerk die gleichen Kommunikationseinstellungen haben. Die Drucktransmitter werden über ein \textit{Write Single Register}-Befehl adressiert. Im Auslieferungszustand hat jeder Drucktransmitter die Adresse "1". Nachdem die neue Adresse in das Adressregister geschrieben worden ist, muss der Drucktransmitter neu gestartet werden. Jetzt kann er unter der neuen Adresse Befehle empfangen. 
Die Drucktransmitter werden nach einander adressiert und in das Netzwerk integriert. 

Die Expansionsventile und der Massenstromsensor werden manuell über ihr Bediendisplay eingestellt. Dann werden  auch sie in das Netzwerk integriert. 
 
Durch System Manager, der in TwinCAT 3 integriert ist,   können die Busklemmen manuell gesteuert werden. Diese Funktion wird genutzt, um die ausgegebene Klemmenspannung bzw. -strom zu prüfen, als auch den korrekten Betrieb der übrigen Komponenten zu prüfen. Nach korrekter Signalverarbeitung über den System Manager kann die Komponente an die SPS angeschlossen werden. Dann wird getestet, ob die Funktionen in dem Programm-Code korrekt das Prozessabbild an die Ausgangsklemme gegeben wird. 

Die Implementierung des Anlagenschutzes folgt. Alle software-seitigen Schutzfunktionen, die zur Sicherheit der Anlage und des Personals dienen und aus der Risikoanalyse hervorgegangen sind, werden getestet. 

Die Regelung der Kälteanlage kann in Betrieb genommen werden. Die Reihenfolge der Inbetriebnahme der Stellsignale war folgende:

\begin{itemize}
\item Anschluss der Heizelemente an die SPS 
\item Anschluss des Drehzahlreglers des Verflüssigungsventilators an die SPS 
\item Anschluss der Schützschalter an die SPS zur Spannungsfreigabe vom Kompressor, Ver-flüssigungs-, Verdampfer- und Abtauverdampferventilator und Schaltung der Magnetventile und des Vierwegeventils
\item Überspielung der Kompressor-Software mit Software für ein externes 4...20 mA-Stellsignal. Dann folgt der Anschluss an die SPS und manuelle Tests per System Manager
\end{itemize}

Bei der Regelung-Inbetriebnahme wurden zunächst die nativen Grenzen für den PID-Regler begrenzt und in kleinen Schritten vergrößert. z. B. bei der Kompressor-Regelung wurde zunächst sowohl das höchste Stellsignal auf 8 mA festgelegt und dann in 2 mA-Schritten bis 20 mA erhöht. Bei dem Kompressor wurde zusätzlich die höchste Frequenz auf 50 Hz begrenzt und erst nach der kältetechnischen Inbetriebnahme (siehe \ref{sec:Inbetriebnahme_KA})) auf die maximalen 70 Hz erhöht. 

Dieses Vorgehen wurde auch beim PID-Regler für den Verflüssigungsventilator angewandt.

Zu guter Letzt wurde die Benutzeroberfläche der Kälteanlage in Betrieb genommen, angepasst und optimiert. 

\section{Fehlersuche}
\label{sec:Fehlersuche}

Nach der gesamten Inbetriebnahme wird der stationäre Kühlbetrieb und  die Funktionalität der Abtaumethoden getestet. Weißt das Anlagenverhalten Unregelmäßigkeiten auf, so muss eine Fehlersuche vorgenommen werden. Zunächst wird der mögliche Fehler in dem Programm-Code gesucht, dann auf der elektrischen Ebene und zu guter letzt auf der der hydraulischen Ebene. Eine Fehlerbehebung auf informationstechnischer Ebene ist meist schnell durchgeführt und ein möglicher Fehler durch Code-Anpassung schnell behoben. Elektrische Fehlersuche kann die Unterstützung von elektrischen Fachpersonal erfordern und hydraulische Fehlerbehebung nur durch Kältetechnik-Fachpersonal durchgeführt werden. 
Im Folgenden werden die Phänome, die Ursachen und die Behebung der folgenden Probleme erläutert:

\begin{itemize}
\item	Ungewollte Kältemittelbewegung im Anlagenstillstand und Kühlbetrieb
\item	Kälteleistungsminderung durch Kältemittelmangel und \textit{Flashgas}
\item 	Schwingender Regelkreislauf
\end{itemize}

\subsection*{Ungewollte Kältemittelbewegung im Anlagenstillstand und Kühlbetrieb}

Vor einem Anlagenstillstand wird ein PumpDown (siehe \ref{subsec:Statusmaschine}) durchgeführt. Das schützt den Kompressor beim Wiederanlaufen vor einer starken Unterkühlung und mögliche Tropfenschläge am Kompressoreingang. Im Sammler ist der Großteil des Kältemittels gelagert und verlässt diesen im Normalfall erst wieder im Betriebsfall. In der aktuellen Anlagenkonfiguration wird das Entweichen des flüssigen Kältemittels durch zwei Rückschlagventile und die zwei Expansionsventile sichergestellt, dargestellt in Abbildung \ref{fig:Problem1}. Es wurde festgestellt, dass die Expansionsventile nicht für diese Funktion geeignet sind. Über einen Stillstand von 12 h findet ein Druckausgleich in der ganzen Anlage statt und das Kältemittel verteilt sich im ganzen Kreislauf.

Die Problem wurde kurzfristig durch das manuelle Schließen der Absperrventile nach dem Massenstromsensor im Anlagenstillstand behoben. Für eine längerfristige Lösung sollte die Installation eine zusätzlichen Magnetventil in die Einspritzleitung geprüft werden. 

Neben der Kältemittelbewegung im Anlagenstillstand, wurde auch eine Kältemittelbewegung im Kühlmodus festgestellt. Läuft die Anlage im Kühlmodus, so ist nach dem Verflüssiger flüssiges Kältemittel anzutreffen. Das flüssige Kältemittel wird nun über die Druckdifferenz in den Abtauverdampfer gedrückt und sammelt sich dort. Dem Kältemittelkreislauf wird so im Kühlbetrieb Kältemittel entzogen. Der Sammler ist nicht mehr gefüllt und es kann zu Massenstromschwankungen kommen, die wiederum in Kälteleistungsschwankungen münden. 
 Auch hier könnte die zusätzliche Installation von Magnetventile Abhilfe schaffen. Weitere Lösungs- und Optimierungvorschläge sind in Abschnitt \ref{cha:Ausblick} zusammengetragen.  


\begin{figure}[htb]
\centering	
	\includegraphics[page=1,width=1.150\textwidth]{Pictures/Inbetriebnahme/Probleme.pdf}
\caption{Kältemittelbewegungen im Anlagenstilland (gelb markierte Ventile). Kältemittelbewegung im Kühlbetrieb in Richtung Abtauverdampfer (grün markierte Ventile).In blau hinterlegte Magnetventile sind mögliche Lösungensvorschläge zur Problembehebung.}
\label{fig:Problem1}
\end{figure}


\subsection*{Geringe Kälteleistung des Luftkühlers}

Im Kühlbetrieb im stationären Zustand sind folgende Punkte aufgefallen: 
\begin{itemize}
\item Hohe Kältemittel-Überhitzung am Ausgang des Verdampfers (20-30 K)
\item Vollständig geöffnetes Expansionsventil
\item Schnelle Vereisung der ersten Rohrlängen im Verdampfer-Wärmeübertrager
\item hoher Druckverlust zwischen Verflüssiger-Ausgang und Expansionsventil-Eingang
\item geringer Massenstrom (20-30 g/s)
\end{itemize}

Mögliche Gründe für dieses abnormale Anlagenverhalten waren :

\begin{itemize}
\item Kältemittelmangel
\item falsch dimensioniertes Expansionsventil
\item \textit{Flash-Gas} vor dem Expansionsventil
\end{itemize}

Zunächst werden 4 kg zusätzliches Kältemittel nachgefüllt. Die gesamte Kältemittelmenge liegt jetzt bei insgesamt 18 kg. Nach einer Überschlagung der Druckverluste zwischen Expansionsventil und Verflüssiger, konnte der Druckverlust als deutlich zu hoch eingestuft werden. Kondensiertes Wasser auf den Rohrleitungen nach dem Trockner, sowie Bläsen im Schauglas wiesen auf \textit{Flash-Gas} hin, vgl. \citep{GAGKG2010}.
Durch einen zu hohen Druckverlust kommt es zur Verdampfung des Kältemittel bevor es durch das Expansionsventil entspannt wird. Durch die Gasbläschen erhöht sich das spezifische Kältemittel-Volumen und verringern gleichzeitig den Massenstrom durch das Expansionsventil. Das Expansionsventil versucht die Überhitzung zu senken, indem es sich öffnet. In diesem Fall versucht das Expansionsventil durch eine vollständige Öffnung einen höheren Massenstrom zu erwirken. Abbildung \ref{fig:Problem_Trockner} zeigt die Zustandspunkte der Anlage im log p,h-Diagramm. Im Diagramm sind der hohe Druckverlust und die hohe Überhitzung mit gelben Pfeilen gekennzeichnet. 

Das Problem konnte durch das Austauschen des Trockners behoben werden. Der Druckverlust zwischen Expansionsventil und Verflüssiger-Austritt verringerte sich von knappen 3 bar auf 0,5 bar. Das Expansionsventil regelte nun mit einer Öffnung von 50-55 $\%$ auf die vorgegebene Überhitzung von 6 K. 


\begin{figure}[h]
\centering		\includegraphics[page= 2,width=1.120\textwidth]{Pictures/Inbetriebnahme/Probleme.pdf}
\caption{Hohe Kältemittel-Überhitzung nach Verdampfer und hoher Druckverlust zwischen Expansionsventil und Verflüssiger dargestellt im log p,h-Diagramm}
\label{fig:Problem_Trockner}
\end{figure}

\begin{figure}[h]
\centering		\includegraphics[page= 1,width=1.05\textwidth]{Pictures/Inbetriebnahme/logphKA.pdf}
\caption{Zustandspunkte der Kältemaschine nach Fehlerbehebung dargestellt im log p,h-Diagramm}
\label{fig:KA_OK}
\end{figure}

\newpage
\subsection*{Schwingender Regelkreislauf}

Da die Expansionsventile nur durch den Modbus RTU ausgelesen werden und nicht gesteuert, können sie nur mir einer PID-Parametereinstellung betrieben werden. Mit den Auslieferungsparameter für den CAREL-PID-Regler wurde im Kühlbetrieb eine Schwingung um den Sollwert der Überhitzung ($\pm$ 4 K) festgestellt. 

Der Grund hierfür ist der örtliche Abstand zwischen Verdampfer-Ausgang und Position des Druck- und Temperaturfühlers vom Expansionsventil (ca. 3 m). Der Temperaturfühler ist, wie in der Praxis üblich, nur ein Oberflächensensor. Der Temperaturfühler erfasst die Temperaturänderung, ausgelöst durch eine Ventilöffnungsänderung,  erst mit einer Verzögerung von 90-120 s. Ein anderes PT100-Element (\textit{Temp$\_$VD$\_$out}), angebracht unmittelbar am Verdampfer-Ausgang,  misst eine Temperaturänderung nach einer Ventilöffnungsänderung bereits nach 10-15 s. 

Eine kurzfristige Lösung des Problems war die Veränderung der Regelparameter des Expansionsventil nach dem Einschwingvorgang bzw. Herunterkühlen der Klimakammer(t = 20 min). Um eine schwingenden Regelkreis im stationären Zustand zu vermeiden, wurden die Parameter manuell umgestellt. 

\begin{table}[htb]
\centering
\caption{Werte für CAREL-PID-Regler für Einschwingvorgang und stationärer Betrieb}\vspace{6pt}
\begin{tabular}{lll}
\hline 
\textbf{PID-Parameter} & \textbf{Einschwingvorgang} & \textbf{Stationärer Betrieb} \\ 
\hline 
\hline
KP & 15 & 5 \\ 
\hline 
Integralzeit & 150 ms & 490 ms \\ 
\hline 
Differentialzeit & 5 s & 5 s \\ 
\hline 
\hline
\end{tabular} 
\label{tab:Regler_Uebersicht}
\end{table}

\section{Kalibrierung Wägesystem}
\label{sec:Waegesystem}

Wie schon im Abschnitt \ref{sec:Waegesystem} erwähnt worden ist, muss das Wägesystem vor jeder Messung kalibriert werden. Für die Kalibierung sind folgende Schritte notwendig:

\begin{itemize}
\item Waagen mit 2 kg vorbelasten
\item Gewichtskalibrierung mit in 500 g Schritten von 500 g bis 2500 g. Anleitung ist in der GUI implementiert
\item Ventilatorkalibrierung
\end{itemize}



\section{Versuche}
\label{sec:Versuche}

Nach der Inbetriebnahme und Behebung der aufgelisteten Defizite werden erste Vereisungs- und Abtauversuche durchgeführt. Der Luftkühler befindet sich in der Klimakammer, die auf 2 °C und 90 $\%$ Luftfeuchtigkeit eingestellt ist und somit als Wärmequelle dient. Nachdem der Luftkühler 1 h vereist worden ist, wird er wieder abgetaut. Hierzu wurde ein Mal die Abtaumethode \textit{elektrisch} und zwei Mal über \textit{Heißgas oben} verwendet. Die Ergebnisse für die Vereisungs- und Abtauphase werden im Folgenden dargestellt und diskutiert. In Tabelle \ref{tab:Versuchsreihe} sind die drei Versuchsreihen mit den dazugehörigen Vereisungszeiten und gewählten Abtaumethoden. 


\begin{table}[htb]
\centering
\caption{Versuchsreihen}\vspace{6pt}
\begin{tabular}{llll}
\hline 
\textbf{Versuchsreihe} & \textbf{Vereisungszeit} & \textbf{Abtaumethode} & \textbf{Position der vier Oberflächen-PT100} \\ 
\hline 
\hline 
I & 47 min  & Heißgas-Oben & Stack-Eingang \\ 
\hline 
II & 53 min  & Heißgas-Oben & Verteilt über unterstes Stack \\ 
\hline 
III & 59 min & Elektrisch & Stack-Eingang \\ 
\hline 
\hline 
\end{tabular} 
\label{tab:Versuchsreihe}
\end{table}


\subsubsection*{Versuchsreihe I: Vereisen}


In den nachfolgenden Diagrammen sind die Temperatur- und Druckverläufe am Luftkühler dargestellt. Das Diagramm \ref{fig:VereisenI} stellt die Verläufe für die Versuchsreihe I dar. Die Druckverläufe von Ein- und Ausgang zeigen einen schwingenden Verlauf mit einer Periodendauer von der gesamten Vereisungszeit. Der Druck am Eingang beträgt durchschnittlich 3,7 bar; am Ausgang durchschnittlich 2,2 bar. Der Druckabfall über den Venturiverteiler und Wärmeübertrager beträgt konstante 1,5 bar über die Vereisungszeit. 

Die Temperaturverläufe sind am Eingang und an den Oberflächensensoren weitegehend konstant und führen wie die Druckverläufe eine Schwingung mit der Periodendauer von ca. 2200 s durch. Am Eingang schwankt die Temperatur in dieser Zeit zwischen 5,9 °C und 7,8 °C. Im Durchschnitt liegt sie bei 7,1 °C. Die Oberflächensensoren betragen im Durchschnitt - 6 °C. Auffällig ist die Temperaturspreizung zwischen den Oberflächensensoren. Zwischen dem Eingang des obersten Einspritzungsrohr (\textit{TEMP$\_$VD$\_$surface$\_$6}) und dem untersten Einspritzungsrohr (\textit{TEMP$\_$VD$\_$surface$\_$2}) beträgt zu jedem Zeitpunkt ca. 1 K. 

Die Eingangstemperatur, die noch vor dem Venturiverteiler gemessen wird, verzeichnet starke periodische Schwankungen in einem kurzen Zeitraum um mehrere Kelvin. Die Schwankungen werden ausgelöst durch die Expansionsventilöffnungsänderung, sprich die Überhitzungsregelung.  Die Schwankungen gilt es zu vermeiden und werden im Abschnitt \textit{Expansionsventil: PID-Parameter-Vergleich} näher diskutiert. 

\begin{figure}[htb]
\centering		\includegraphics[page=1,width=1.08\textwidth]{Pictures/Inbetriebnahme/Abtaumethoden_Tempverlaufe_Vereisen.pdf}
\caption{Versuchsreihe I: Temperaturverläufe am Luftkühler (VD) während Vereisungszeit}
\label{fig:VereisenI}
\end{figure}


\subsubsection*{Versuchsreihe II: Vereisen}

Die Versuchsreihe II zeigt von Versuchsreihe I unterschiedlichen Temperatur- und Druckverlauf. Der Druckverlauf am Ein- und Ausgang zeigt einen instationären Verlauf bis zur Zeit von 2200 s. Danach ist ein stationärer Zustand erreicht. Im stationären Zustand werden Druckwerte wie in der Versuchsreihe I erreicht. Der instationäre Verlauf spiegelt sich auch in den Temperaturverläufen wieder. Die Instationärität wird hervorgerufen durch ein sehr langsam schließendes Expansionsventil. Die Regler der Überhitzungsregelung ist in dieser Versuchsreihe deutlich träger eingestellt als noch bei der Versuchsreihe I. 

Ab 750 s sind sprunghafte Temperaturänderungen an den Oberflächensensoren zu erkennen. Mögliche Ursachen hierfür ist die einsetzende Abtauung der Klimakammer und Türöffnungen während des Versuchs und sind bei zukünftigen Versuchen zu vermeiden. Zum Ende der Vereisungsphase, im stationären Zustand, treten im Verdampfereingang erneut periodische Temperaturschwankungen auf. 

\begin{figure}[htb]
\centering		\includegraphics[page=2,width=1.08\textwidth]{Pictures/Inbetriebnahme/Abtaumethoden_Tempverlaufe_Vereisen.pdf}
\caption{Versuchsreihe II: Temperaturverläufe am Luftkühler (VD) während Vereisungszeit}
\label{fig:VereisenII}
\end{figure}


\subsubsection*{Versuchsreihe III: Vereisen}

Die Versuchsreihe III zeigt wie schon Versuchsreihe II zunächst einen instationären und später, ab 1800 s, einen stationären Verlauf. Erneut ist ein träg eingestellter PID-Regler für die Überhitzungsregelung der Grund für dieses Verhalten. Die Oberflächensensoren messen im Durchschnitt -5,5 °C. Die Eingangs- und Ausgangstemperatur liegen im Durchschnitt für die stationäre Phase bei 7 °C und -3,5 °C. Der Druck am Eingang beträgt 3,7 bar im Durchschnitt und der Ausgangsdruck liegt im Durchschnitt 2,2 bar.  


\begin{figure}[htb]
\centering		\includegraphics[page=3,width=1.08\textwidth]{Pictures/Inbetriebnahme/Abtaumethoden_Tempverlaufe_Vereisen.pdf}
\caption{Versuchsreihe III: Temperaturverläufe am Luftkühler (VD) während Vereisungszeit}
\label{fig:VereisenIII}
\end{figure}


\newpage
\subsubsection*{Versuchsreihe I: Abtauen mittels \textit{Heißgas-Oben}}

Das Diagramm \ref{fig:I_Heissgas_oben}(a) zeigt den Verlauf der Temperatursensoren am Luftkühler über die Abtauzeit. Zu Beginn des Abtauens wird ein Pumpdown durchgeführt. Dann wird das Vierwegeventil geschaltet, der Kompressor läuft an und heißes Gas fängt an durch den Verdampfer zu strömen. Alle Temperaturen steigen in der Abtauzeit von 2 min simultan an. Die maximale Temperatur von 43,3 °C werden am Ende der Abtauzeit am Ein- und Ausgang des Verdampfers erreicht. 
Am Eingang der Wärmetauscherstacks werden maximal 34,8 °C erreicht. Am obersten Stackeingang (\textit{Temp$\_$VD$\_$surface6}) werden nur 29,5 °C erreicht. 

Nach der Abtauzeit folgt die Abtropfzeit von 10 min. Alle Temperaturen fallen über diesen Zeitraum. Die höchste Temperatur ist am Eingang des Verdampfers zu verzeichnen. Die Oberflächentemperaturänderung werden kleiner und die Temperaturen streben gegen einen Grenzwert. (\textit{Temp$\_$VD$\_$-surface6}), positioniert am obersten Wärmeübertrager-Eingang, bleibt nach 550 s bei konstanten 10,5 °C. Ein möglicher Grund hierfür ist die natürliche Konvektion innerhalb des Wärmeübertrager während der Abtropfzeit. Die abgegebene Wärme steigt nach oben und führt so zu einer Temperaturschichtung. Die unteren zwei Oberflächensensoren (\textit{Temp$\_$VD$\_$surface2}~ und  \textit{Temp$\_$VD$\_$surface4}) messen am Ende der Abtropfzeit, bei 810 s, eine Temperatur von 2,3 °C. Zwischen dem obersten und untersten Oberflächentemperatursensor ist eine Temperaturspreizung von 8 K zu verzeichnen. 

\ref{fig:I_Heissgas_oben}(b) zeigt die Massenverläufe und die Öffnung der Expansionsventils über die gesamte Abtauphase. Mit dem PumpDown sinkt die Masse des Verdampfer, weil das noch vorhandene Kältemittel abgesaugt wird und der Verdampfer an Gewicht verliert. Mit Beginn der Abtauung durch Heißgas steigt sofort die Masse des Verdampfers. Innerhalb der 2 min steigt die Verdampfermasse um ca. 3,8 kg an. Nach dem Beginn der Abtropfphase sinkt die Verdampfermasse erneut, da das Eis geschmolzen ist und nun über den Abfluss abgeführt wird. Die Abtropfmasse steigt nun stetig an und der Verdampfer wird wieder leichter. In der Abtropfphase wird der Verdampfer um 910 g leichter. Die gesamte Abtropfmasse beträgt 2456 g. 


Nach 600 s Abtropfzeit beginnt die Kälteanlage wieder mit dem Kühlmodus. Alle Temperatur sinken, der Zyklus beginnt erneut. 



\begin{figure}%[h]
\centering
\subfigure[Temperaturverläufe am Luftkühler. Oberflächen-Temperatursensoren(\textit{surface}) sind angebracht am Einspritzrohr]
{\includegraphics[page=1,width=1.03\textwidth]{Pictures/Inbetriebnahme/Abtaumethoden_Tempverlaufe.pdf}}
\subfigure[Verläufe von der Abtropfmasse und Verdampfermasse, Öffnung des Expansionsventils aufgetragen über die Zeit]{\includegraphics[page=21,width=1.08\textwidth]{Pictures/Inbetriebnahme/Diagramme_Versuche.pdf}}
\caption{Versuchsreihe I: Abtauen mittels \textit{Heißgas-Oben}}
\label{fig:I_Heissgas_oben}
\end{figure}

\subsubsection*{Versuchsreihe II: Abtauen mittels \textit{Heißgas-Oben}}

In der Versuchsreihe II wird erneut mittels \textit{Heißgas-Oben} abgetaut. Die Oberflächentemperatursensoren werden für diese Messung an nur einem Wärmeübertragungsrack angebracht. Die Oberflächentemperatursensoren sind PT100-Elemente. Ein Rack besteht aus 6 x 6 Rohrleitungen, durch die das Kältemittel geführt wird und Wärme aus der Umgebung aufnimmt. Die Oberflächensensoren sind an folgenden Rohren angebracht: 1. Rohr(Einspritzrohr, \textit{Temp$\_$VD$\_$surfaceX12}), 12. Rohr (\textit{Temp$\_$VD$\_$surfaceX12}), 25. Rohr (\textit{Temp$\_$VD$\_$surfaceX25}) und 36.  Rohr ((\textit{Temp$\_$VD$\_$surfaceX36})).

Die höchste Temperatur wird am Verdampfereingang mit 43,9 °C erreicht. Die Temperaturverläufe zeigen ähnliche Charakteristika wie die Temperaturverläufe aus der Versuchsreihe I. Nachdem das Heißgas 2 min durch den vereisten Wärmeübertrager geströmt ist, misst jeder Sensor die örtliche maximale Temperatur.  Danach fallen alle Temperaturwerte mit ähnlicher Geschwindigkeit wie sie zuvor gestiegen waren. Nach 2 min flachen die Temperaturkurven ab und sinken mit geringer Geschwindigkeit während der Abtropfphase weiter. Die Temperaturspreizung zwischen den Oberflächentemperaturen beträgt ca. 1,5 K. Die Oberflächentemperatur am 36. Rohr ist die wärmste und die anderen drei Oberflächentemperaturen sind nahezu gleich. Die Oberflächentemperatur am 12. Rohr \textit{Temp$\_$VD$\_$surfaceX12}) steigt nach Beginn der Heißgas-Abtauung, jedoch nicht so stark wie die anderen Oberflächentemperaturen. Die maximale Temperatur wird 60 s nach dem Ende des Abtauvorgangs, bei 310 s, mit 8 °C erreicht. Der Oberflächensensor hat 

Nach der Abtropfdauer startet der Kühlmodus und die Temperaturen fallen wieder. 

\begin{figure}[htb]
\centering		\includegraphics[page=2,width=1.08\textwidth]{Pictures/Inbetriebnahme/Abtaumethoden_Tempverlaufe.pdf}
\caption{II: Abtauen mittels \textit{Heißgas-Oben}}
\label{fig:II_Heissgas_oben}
\end{figure}

\subsubsection*{Versuchsreihe III:  \textit{Elektrisches} Abtauen}

Nachdem der Verdampfer vereist worden ist, wird der Verdampfer mittels einer elektrischen Heizung abgetaut. Die elektrische Heizung umfasst 20 Heizstäbe, die in dem Wärmeübertrager integriert sind. Ein Heizstab hat eine maximale Leistung von 220 W. Die Bodenplatte des Luftkühlers ist mit 3 Heizstäben á 625 W ausgestattet. Die gesamte elektrische Abtauleistung beträgt 6,275 kW. Die Heizelemente sind stufenlos zwischen 0$\dots$10 V regelbar.  

In diesem Versuch wurde mit der maximalen Leistung, sprich 6,275 kW, abgetaut. Abbidlung \ref{fig:III_Elektrisch} zeigt die Temperatursensoren des Luftkühlers dar. 

Wie auch bei den Versuchsreihen I und II wird auch hier zunächst ein Pumpdown durchgeführt. Danach fängt die Abtauphase an und die Temperaturen steigen. 

\begin{figure}[htb]
\centering		\includegraphics[page=3,width=1.08\textwidth]{Pictures/Inbetriebnahme/Abtaumethoden_Tempverlaufe.pdf}
\caption{III: \textit{Elektrisch} Abtauen mit 6,2 kW Heizleistung}
\label{fig:III_Elektrisch}
\end{figure}

%\newpage
\subsubsection*{Expansionsventil: PID-Parameter-Vergleich}
\label{subsubsec:EV}

Für eine hohe Reproduzierbarkeit der Versuche ist ein stationärer Zustand nötig. Ein instationäres System verringert die Aussagekraft über die späteren Versuchsergebnisse. Das Expansionsventil hat großen Einfluss auf die angestrebte Stationarität der Kälteanlage. Aus diesem Grund ist der Verlauf der Überhitzung über die Zeit für alle drei Versuchsreihen aufgezeichnet. 

Das Expansionsventil kann zu entweder mit den PID-Parameter für den stationären Betrieb oder für den Einschwingvorgang programmiert sein. Regelt das Expansionsventil von Beginn an mit stationären PID-Parameter, so hat dies einen sehr langen Einschwingvorgang zur Folge (siehe Kurve \textit{Stationäre PID-Parameter} im Diagramm \ref{fig:EV_Vergleich}).

Ist das Expansionsventil mit den Einschwing-PID-Parameter programmiert, wird der Sollwert schnell erreicht, jedoch schwang die Überhitzung periodisch um den Sollwert mit Abweichungen um $\pm 3$ K (siehe Kurve \textit{Einschwing-PID-Parameter} im Diagramm \ref{fig:EV_Vergleich}). 

Werden die PID-Parameter nach Erreichen des Sollwertes geändert, so verkürzt sich die Einschwingphase und die Schwingungsamplitude im stationären Zustand (siehe Kurve \textit{Einschwing + Stationäre PID-Parameter} im Diagramm \ref{fig:EV_Vergleich}). Für zukünftige Versuche wird im Abschnitt \ref{cha:Ausblick} Optimierungsmöglichkeiten aufgezeigt. 

\begin{figure}[htb]
\centering		\includegraphics[page=1,width=1.08\textwidth]{Pictures/Inbetriebnahme/EV_Vergleich.pdf}
\caption{}
\label{fig:EV_Vergleich}
\end{figure}



\subsection{Fehlerabschätzung}
\label{subsec:Fehler}

