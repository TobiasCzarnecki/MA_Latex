\chapter{Ausblick und Optimierungsvorschläge}
\label{cha:Ausblick}
Um qualitative Aussagen bezüglich der Abtaueffizienz und -geschwindigkeit treffen zu können, werden weitere Messungen durchzuführen sein. Zunächst wird vorgeschlagen, den Referenzumgebungspunkt bei 2 °C und 90$\%$ Luftfeuchtigkeit zu belassen und mit einer Vereisungszeit von 1 h alle Abtaumethoden hinreichend oft zu vermessen. Erst dann wird ein Vergleich der Abtaumethoden mittels der gewonnen Daten möglich sein. 

Um die Performance der Kälteanlage, der SPS und des Wägesystems weiter zu optimieren, wurde eine Liste von Punkten zusammengetragen, die zu einer Verbesserung des gesamten Prüfstands führen. Die Punkte haben sich während der Inbetriebnahme und bei der Durchführung der ersten Messungen ergeben. Die Optimierungsmaßnahmen haben unterschiedlich hohe Prioritäten und werden unterschiedlich stark zur Verbesserung des Prüfstandes führen.

Die Prioritätseinschätzung bezieht sich auf Sicherheit, Funktionalität und Messerfassung der Kälteanlage. Der Aufwand betrachtet Arbeitsstunden, Einarbeitungszeit und Kosten der Maßnahmen. Alle Einschätzungen sind subjektiv und bedürfen die Absprache mit dem Anlagenbetreiber.


\section*{Hohe Priorität}

 \textbf{Kältemaschine}

\begin{itemize}
\item die Wiederinbetriebnahme der mechanischen Sicherheitskette des Bock-Kompressors durch Austausch des Funktionsmoduls am EFC-Frequenzumrichter. Weitere Hardware- Sicherheitsorgane wie mechanische Druckschalter erhöhen das Sicherheitsniveau der Anlage.
\end{itemize}	

\textbf{Expansionsventil}
	\begin{itemize}
	 \item Den Temperatursensor näher an den Verdampfer-Austritt positionieren und/oder externes analoges Signal für Druck- und Temperaturwerte  verwenden. Hierfür \textsc{Carel}-Sensoreingangs-kanäle S3 und S4 benutzten, da diese noch nicht belegt sind. Analoge Ausgangsklemmen, EL 4024 für 4\dots 20 mA oder EL 4008 für 0\dots 10 V, von der SPS besitzen noch unbenutzte Kanäle. \textsc{Carel}-Druck und Temperatursensoren (auf Anschlüssen S1 und S2) als \textit{Back-up}-Regelung konfigurieren.
	 \end{itemize}
	 
	
\textbf{Modbus RTU}
\begin{itemize}

\item Baudrate von \textit{Modbus COM} auf 115 200 bps hochstellen. Dies ermöglicht eine schnellere Abfrage der Slaves. Bisher dauert ein Datenerfassungszyklus 2-4 s lang. Anpassung der Baudrate führt zu einer Verringerung des Datenerfassungszyklus auf  unter 1 s. Achtung: Störsignale werden höher auf der Datenleitung. Es wird empfohlen die Timeout zwischen Modbus-\textit{Request} und -\textit{Response} hochzustellen.
\end{itemize}	 

\textbf{Informationstechnik}
Optimierung des Programmcodes und Implementierung weiterer \textit{Features}. 
\begin{itemize}
\item Datenspeicherung der Globalen Variablen per .CSV- Datei oder direkte Übermittlung der Daten in die Datenbank MySQL programmieren.
\end{itemize}

\textbf{Wägesystem}

\begin{itemize}
\item Reduzierung der Waagenanzahl  auf drei Waagen. Durch Fixierung von der linken Seite wird es kein Kippmoment bei Betrieb der Ventilatoren und Klimakammer geben. Ein Wägesystem mit drei Waagen hat weniger Freiheitsgrade als ein System mit vier Waagen.

\item  Kältemittel-Gewichtserfassung: Berechnung oder praktische Erfassung der Kältemittelmasse im Luftkühler, um errechnete Eis-Summe von Öffnung des Expansionsventiles zu entkoppeln.
\end{itemize}

\section*{Mittlere Priorität}

	\textbf{Expansionsventil}: 
	\begin{itemize}
	 \item Die Reglerparameter sollten optimiert werden. Entweder kann hierfür das \textit{Autotuning}- Verfahren von \textsc{Carel} benutzt werden oder das Ziegler-Nichols-Verfahren, dass per SPS ausführt werden kann. Für das Ziegler-Nichols-Verfahren gibt es in der OSCAT-Bibliothek die Funktionsbausteine \textit{CONTROL$\_$SET1} oder \textit{CONTROL$\_$SET2}. Die Funktionsbausteine, führen nach Eingabe der kritischen Verstärkung und Periodendauer, das Verfahren automatisiert durch.
	 \item 	Die Einspritzdüse tauschen oder weiteres Magnetventil installieren, um einen Druckausgleich im Anlagenstillstand zu verhindern. Nach einem Pumpdown bleibt das flüssige Kältemittel im Sammler.
	\end{itemize}	

\textbf{Abtauverdampfer}: 
	\begin{itemize}
\item  Heizstäbe \newline
Das flüssig, im Abtauverdampfer gefangene Kältemittel  wird bei T= 35 °C  verdampfen und der Druck sich auf ca. 9 bar erhöhen. Das Kältemittel wird zum Austritt über die Rückstoßventile aus dem Verdampfer gezwungen und kein flüssiges Kältemittel sammelt sich im Kühlbetrieb im Abtauverdampfer.


\item Magnetventile \newline
 Rückschlagventile durch Magnetventile ersetzten und über ein Schaltschütz öffnen und schließen. Schaltschütz könnte über ein digitales Ausgangssignal über die SPS geschalten werden. Zusätzliches Signal in die Statusmaschine implementieren. 

\item Umlenkung der warmen Verflüssiger-Abluft \newline
Verflüssiger-Ventilator mit niedrigeren Drehzahlen betreiben und Umlenkung der warmen Luft in den Wärmeübertrager des Abtauverdampfers. Erhöhung der Temperatur führt zum Verdampfen des flüssigen, eingeschlossenen Kältemittels. 
\end{itemize}	

\textbf{Regelung}
	\begin{itemize}
	\item Nach Optimierung der Regelperformance des Expansionsventils kann die gesamte Regelung der Kälteanlage optimiert werden. Hierzu die gängigen empirischen Optimierungsverfahren (\textit{Ziegler-Nichols-Verfahren}) anwenden, um bessere Regelparameter für die PID-Regler zu ermitteln. Die Standardabweichung wird gesenkt und dadurch eine geringere Abweichung von den gewünschten Oberflächentemperaturen im Verdampfer erreicht. 
	\end{itemize}

\section*{Niedrige Priorität}

\textbf{TwinCat 3} 
\begin{itemize}
\item Datenspeicherung der Globalen Variablen per .CSV- Datei oder direkte Übermittlung der Daten in die Datenbank MySQL programmieren. 
\item Meldungslog einbauen, indem Fehler und aktuelle Statusmeldungen geschrieben werden können. Erleichtert die Überwachung und Auswertung der Messdaten.
\item Abtauablauf: Einschalten der Bodenheizung mittels TIMER verzögern. Bei Heißgasabtauung die Abtaudauer optimieren bzw. an die vorhandene Eismenge anpassen. 
\end{itemize}

\textbf{GUI}

\begin{itemize}
\item Erweiterung der Steuerungselemente-Bibliothek um folgende Kältekomponenten: Wärmeübertrager, Magnetventil, Vierwegeventil, Kompressor. Übersichtlichere und eine intuitiv verständlichere Benutzeroberfläche wäre das Ergebnis.
\item Messdaten-Aufzeichnungsfeld in die Benutzeroberfläche integrieren, siehe Konzept von HIL2-Benutzeroberfläche in \citep{Nuerenberg2015}.
\end{itemize}

\textbf{Wägesystem}
\begin{itemize}
\item Optimierung der Schmelzwasser-Wägung bzw. des -Ablaufes für den automatisierten Ver-eisungs- und Abtauungsprozess. Zurzeit ist manuelles Leeren nach jeder Messreihe nötig.
\end{itemize}


