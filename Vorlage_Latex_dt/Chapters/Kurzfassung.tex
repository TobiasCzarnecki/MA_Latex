Eine Vereisung des Verdampfers in einem Kältekreislauf führt zu einer Minderung des übertragenen Wärmestroms sowie des Anlagenwirkungsgrads. Der vereiste Wärmeübertrager muss daher von Zeit zu Zeit abgetaut werden. Dafür werden in der Praxis sowohl elektrische als auch Heißgas-Abtaumethoden eingesetzt.

Zur genaueren Untersuchung dieser Strategien wird ein experimenteller Aufbau eines Kältekreislaufes mit austauschbarem Luftkühler modifiziert. Im Kältekreislauf installierte Temperatur- und Drucksensoren sowie eine elektrische Leistungsmessung erlauben die energetische Bilanzierung aller einzelnen Komponenten des Kältekreises.

Ein bereits vorhandener Wägeaufbau für die Messung der Massenänderung des Luftkühlers wird optimiert. Ein Konzept für einen mobilen Aufbau zur Untersuchung verschiedener Prüflinge wird erstellt und ein Kalibrierungsverfahren für den optimierten Wägeaufbau entwickelt. Ziel ist die Messung der zeitlich veränderlichen Eis- bzw. Tauwassermenge im Luftkühler im Normal- bzw. Abtaubetrieb sowie die Veränderung des 2D-Schwerpunktes des Luftkühlers. Das Auslesen der Messdaten erfolgt automatisiert.

Für den Kältekreislauf wird softwareseitig ein Steuerungskonzept entworfen und mittels einer speicherprogrammierbaren Steuerung der Fa. Beckhoff umgesetzt. Die SPS ermöglicht einen vollautomatisierten Betrieb des Kältekreislaufs nach Nutzervorgabe sowie das Auslesen aller Sensoren und die Darstellung der Messwerte. Ein softwareseitiger Anlagenschutz inklusive Funktionstest wird in der SPS vorgesehen. Bei Bedarf erfolgt eine Anpassung der Regelparameter.
 
Nach der Inbetriebnahme des gesamten Systems wird ein Luftkühler in einer Klimakammer unter verschiedenen Randbedingungen vermessen. Neben der Auswertung der Messergebnisse erfolgen eine Bewertung der Messungen hinsichtlich ihrer Reproduzierbarkeit sowie eine Abschätzung der Messfehler.

??? Diese Kurzzusammenfassung hat 232 Wörter 
