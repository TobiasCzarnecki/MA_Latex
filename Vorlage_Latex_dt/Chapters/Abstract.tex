Icing of the evaporator in a refrigeration cycle leads to an impairment of the transferred heat flow and the system efficiency. The icy heat exchanger must therefore be defrosted from time to time.  Electrical and hot gas defrosting methods are used in practice.

For more detailed study of these strategies an experimental setup of a refrigeration circuit with replaceable air cooler is modified. In the refrigeration cycle installed temperature and pressure sensors and an electrical power measurement allows the energy balance of all components of the refrigerant circuit.

An existing scale system for measuring the mass of the ice within the air cooler is optimized. A concept for a mobile setup to study different samples will be created and developed. A calibration method for the optimized scale system will be developed and implemented into the programmable logic controller(PLC). The aim is to measure the time-varying amount of ice and condensation in the air cooler in normal cooling or defrosting mode.  Also the effect on the change of center of gravity (2D) of the air cooler will be analyzed. The reading of sensor data is automated and will be caried out by the PLC.

For the refrigeration cycle, a control concept is software-designed and implemented by a PLC of the company Beckhoff. The PLC enables fully automated operation of the refrigeration cycle by user default, the reading of all sensors and the display of measured values. A software-governed system protection, including function test is also provided in the PLC. If necessary, an adjustment of the control parameters will be caried out.
 
After commissioning of the entire system, an air cooler is measured in a climate chamber under different boundary conditions. In addition to the evaluation of the measurement results carried out an evaluation of the measurements in terms of their reproducibility and an estimate of the measurement error is made.

