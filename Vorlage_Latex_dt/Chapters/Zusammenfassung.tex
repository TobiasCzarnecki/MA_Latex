\chapter{Zusammenfassung}
\label{cha:Zusammenfassung}

In dieser Arbeit wurde ein Prüfstandsaufbau für die Untersuchung von Vereisungs- und Abtauvorgänge von Luftkühler optimiert. Der Prüfstandsaufbau sollte alle kältetechnischen Komponenten bilanzieren können und den zeitlichen Verlauf der entstehenden Eismenge  bestand aus wurde in vier Phase unterteilt. 

In der ersten Phase wurde ein Wägesystem zur Messung der Eismenge und dessen 2D-Schwerpunktes in einem Luftkühler konzipiert und gebaut. Das Wägesystem besteht aus vier Auflagerpunkten jeweils versehen mit einer Blattfedern. Die Blattfeder fungiert mit einer Waage in einer Parallelschaltung. Ein Kalibrierungsverfahren wurde entwickelt, um vom gemessenen Waagengewicht auf die tatsächliche Belastung im Luftkühler schließen zu können.
Die vier Waagen messen im Verbund und ermitteln die gesamte Gewichtsänderungen im Luftkühler in Echtzeit. 

In der zweiten Phase wurde mittels einer SPS der Fa. Beckhoff und der Software TwinCat 3 ein Programm-Code zur Steuerung der Prüfstandes und Auslesung aller installierten Sensoren entworfen. Der Programm-Code ist nach dem Konzept einer Statusmaschine entwickelt worden. Die Waagen werden über eine RS232-Kommunikation ausgelesen. Zwei Modbus RTU- System kommunizieren mit acht Drucksensoren, zwei Expansionsventile und einem Massenstromsensor über zwei RS485-Netzwerke. Mit zwölf PT100-Elementen und den Messdaten der anderen Sensoren ist eine komplette energetische Bilanzierung der Kältetechnikkomponenten in Echtzeit möglich. 
Die Statusmaschine wird mittels Transitionsvariablen in Kühl-, Abtau-, Leerlauf oder Pumpdown-Modus versetzt. Ein soft- und hardwareseitiger Anlagenschutz sorgt für Sicherheit für Benutzer und Maschine. 

In der dritten Phase wurde die Kälteanlage inklusive ihres Regelkonzeptes in Betrieb genommen. In der dieser Phase wurden mehrere Mängel festgestellt und behoben. Dem Expansionsventils war es nicht möglich trotz kompletter Öffnung die Überhitzung nach dem Verdampfer zu senken. Ein niedriger Massenstrom von ca. 20-30 g/s führte im Betrieb nicht zur erwünschten Kälteleistung des Luftkühlers (Auslegungsmassenstrom 77 g/s). Mehrere Maßnahmen wurden durchgeführt, um die Kälteleistung des Luftkühlers zu steigern. 4 kg Kältemittel wurden nachgefüllt und die Kältemittelrohre mit Isolierung versehen.  Dann wurde der Kältemitteltrockner ausgetauscht. Der alte, verstopfte Trockner hatte zu einem \textit{Flashgas}-Phänomen in der Einspritzleitung geführt. Verstopfte Poren führten zu einem Druckverlust von 4 bar. Spontane Verdampfung führte nach dem Trockner zur Gasbildung und verringerte den Massendurchfluss durch das Expansionsventil. 

In der letzten und vierten Phase der Arbeit wurde eine Bedienoberfläche zur Steuerung der Kälteanlage entworfen, eine Optimierung der Regelparameter  und die ersten Abtauversuchen durchgeführt. Die Bedienoberfläche ermöglicht eine sicher Bedienung der Kälteanlage durch einen Benutzer. Die Optimierung der Regelparameter führte zu der Erkenntnis, dass die Sensoren für das Expansionsventile örtlich zu weit von dem Ausgang des Verdampfers  entfernt sind. Damit haben die Expansionsventile eine sehr hohe Reaktionszeit, welche zu Schwingungen in der gesamten Kälteanlage führte. Eine Änderung der PID-Paramter des Expansionsventils auf KP = 5, TN = 490 ms und TV = 5 s  wurde die regelungstechnische Dämpfung der Kälteanlage erhöht. Diese Einstellungen führen zu einem langsameren Einschwingprozess, verhindern jedoch einen schwingenden Regelkreis. 

Erste Vereisungsversuche, mit einer Dauer von 1 h, wurden in einer Klimakammer durchgeführt. Die Umgebungsbedinungen wurden auf 2 °C und 90 $\%$ Luftfeuchtigkeit gesetzt.
Die Ergebnisse der ersten Versuche offenbaren weitere Optimierungsmöglichkeiten. Die direkte Steuerung der Expansionsventile durch die SPS mittels analogen Ausgangssignalen ist zu prüfen.  Eine Implementierung zweier Reglereinstellungen für  "Einschwingvorgang" und "Stationärer Kühlbetrieb" für das Expansionsventil wird empfohlen und zu prüfen sein. 

 

 