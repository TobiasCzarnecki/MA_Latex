\chapter{Zusammenfassung}
\label{cha:Zusammenfassung}


In dieser Arbeit wurde ein Prüfstandsaufbau für die Untersuchung von Vereisungs- und Abtauvorgängen von Luftkühlern optimiert. Der Prüfstandsaufbau kann alle kältetechnischen Komponenten bilanzieren und den zeitlichen Verlauf der entstehenden Eismenge bestimmen. In vier Phasen wurde der Aufbau durchgeführt. 

In der ersten Phase wurde ein Wägesystem zur Messung der Eismenge in einem Luftkühler konzipiert und gebaut. Das Wägesystem besteht aus vier Auflagerpunkten, die jeweils mit einer Blattfeder versehen sind. Die Blattfeder fungiert mit einer Waage in einer Feder-Parallelschaltung. Ein Kalibrierungsverfahren wurde entwickelt, um vom gemessenen Gewicht auf die tatsächliche Belastung im Luftkühler schließen zu können.

In der zweiten Phase wurde mittels einer SPS der Fa. Beckhoff und der Software TwinCat 3 ein Programm-Code zur Steuerung des Prüfstandes und der Auslesung aller installierten Sensoren entworfen. Der Programm-Code ist nach dem Konzept einer Statusmaschine entwickelt worden. Die Waagen werden über eine RS232-Kommunikation ausgelesen. Zwei Modbus RTU-System kommunizieren mit acht Drucksensoren, zwei Expansionsventilen und einem Massenstromsensor über zwei RS485-Netzwerke. Mit zwölf Pt100-Elementen und den Messdaten der anderen Sensoren ist eine komplette energetische Bilanzierung der Kältetechnikkomponenten in Echtzeit möglich. 
Die Statusmaschine wird mittels Transitionsvariablen in Kühl-, Abtau-, Leerlauf oder Pumpdown-Modus versetzt. Ein soft- und hardwareseitiger Anlagenschutz sorgt für Sicherheit für Benutzer und Maschine. 

In der dritten Phase wurde die Kälteanlage inklusive ihres Regelkonzeptes in Betrieb genommen. Mehrere Mängel wurden festgestellt und behoben. Ein niedriger Massenstrom von ca. 20-30 g/s sowie eine hohe Überhitzung (20- 30 K) führten im Betrieb nicht zur erwünschten Kälteleistung des Luftkühlers. Mehrere Maßnahmen wurden durchgeführt, um die Kälteleistung des Luftkühlers zu steigern. 4 kg Kältemittel wurden nachgefüllt und die Kältemittelrohre wurden mit Dämmung versehen.  Der alte, verstopfte Trockner hatte zu einem \textit{Flashgas}-Phänomen in der Einspritzleitung geführt. Verstopfte Poren führten zu einem Druckverlust von 4 bar. Durch spontane Verdampfung kam es hinter dem Trockner zu Gasbildung, was den Massendurchfluss durch das Expansionsventil verringerte. 

In der letzten und vierten Phase der Arbeit wurde eine Bedienoberfläche zur Steuerung der Kälteanlage entworfen und eine Optimierung der Regelparameter sowie die ersten Abtauversuchen durchgeführt. Die Bedienoberfläche ermöglicht eine sichere Bedienung der Kälteanlage durch einen Benutzer. Die Optimierung der Regelparameter führte zu der Erkenntnis, dass die Oberflächensensoren für das Expansionsventil örtlich zu weit vom Ausgang des Verdampfers entfernt waren. Die örtliche Entfernung und die träge Wärmeübertragung durch das Rohr hin zum Oberflächensensor führen zu einer zeitlichen Verzögerung.  Damit haben die Expansionsventile eine sehr lange Reaktionszeit, welche zu periodischen Über- und Unterschwingungen des Sollwertes führte. Durch eine Änderung der PID-Parameter des Expansionsventils wurde die regelungstechnische Dämpfung der Kälteanlage erhöht. Diese Einstellungen führen zu einem langsameren Einschwingprozess, verhindern jedoch einen schwingenden Regelkreis im stationären Betrieb. Die direkte Steuerung der Expansionsventile durch die SPS mittels analoger Ausgangssignale ist zu prüfen. Eine Implementierung zweier Reglereinstellungen für \textit{Einschwingvorgang} und \textit{Stationärer Kühlbetrieb} für das Expansionsventil wird empfohlen.

Erste Vereisungsversuche wurden in einer Klimakammer durchgeführt. Die Umgebungsbedingungen wurden auf 2 °C und 90 $\%$ Luftfeuchtigkeit gesetzt. Konstante Oberflächentemperaturen auf dem Wärmeübertrager des Verdampfers  wurden über den Zeitraum verschiedener Vereisungsversuche nachgewiesen. Über den Versuchszeitraum konnte die Reproduzierbarkeit einer konstanten Eismenge ($\pm 300 g$)  hinreichend bestätigt werden. Weitere Optimierungsmöglichkeiten, siehe Abschnitt \ref{cha:Ausblick}, wurden identifiziert und werden die Reproduzierbarkeit weiter erhöhen. Im Zusammenspiel mit der Klimakammer wurden konstante Randbedingungen für das Kristallwachstum auf der Verdampferoberfläche ermöglicht. 
Die durchgeführten Abtauversuche weisen unterschiedliche Verläufe der Temperaturen auf dem Verdampfer und des Schmelzwassers auf. Die Abtauung mittels Heißgas verläuft schneller und auf höherem Temperaturniveau als die elektrische Abtauung. Aufgrund mangelnder Versuche ist noch keine Aussage über die Energieeffizienz der Abtaumethoden möglich. Jedoch sind alle Voraussetzungen für die energetische Bilanzierungen der Abtaumethoden erfüllt und können in weiteren Versuchen ermittelt werden. 


 

 