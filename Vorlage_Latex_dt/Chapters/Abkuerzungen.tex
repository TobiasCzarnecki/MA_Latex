\chapter*{Nomenklatur}
\begin{onehalfspacing}
\begin{longtable}[h]{p{0.15\textwidth} p{0.65\textwidth} p{0.1\textwidth}}
		\caption*{\textbf{Formelzeichen und Einheiten}} \\
		\\
		\textbf{Symbol} & \textbf{Bedeutung} & \textbf{Einheit} \\ %\hline 
		\endhead
		\\
		\multicolumn{3}{c}{Fortsetzung auf der nächsten Seite} \\
		\endfoot
		\multicolumn{3}{c}{ } \\
		\endlastfoot
		
		$A$ & Fläche & \squaremetre\\
		$c_{p}$&spezifische Wärmekapazität bei konstantem Druck&\joule\per(\kilogram\usk\kelvin)\\
		$c_{E}$&spezifische Wärmekapazität von Eis&\joule\per(\kilogram\usk\kelvin)\\		
		$c_{F}$&spezifische Wärmekapazität von flüssigem Wasser&\joule\per(\kilogram\usk\kelvin)\\
		$c_{pL}$&spezifische Wärmekapazität von Luft&\joule\per(\kilogram\usk\kelvin)\\	
		$c_{pD}$&spezifische Wärmekapazität von Wasserdampf &\joule\per(\kilogram\usk\kelvin)\\		
		%$C$&Wärmekapazität&\watt\per\kilogram\\
		$E$ & Elastizitätsmodul & N \per m\squaremetre \\
		%$E$ & Exergie & \joule\\
		$E$ & Energie & \joule\\
		%$e$ & spezifische Exergie & \joule\per\kilogram\\
		$g$ & Schwerebeschleunigung & \meter \per \second$^2$\\
		$h $ & spezifische Enthalpie & k\joule \per \kilogram\\		
		$h_{K} $ & spezifische Kondensationsenthalpie & k\joule \per \kilogram\\	
		$h_{S} $ & spezifische Schmelzenthalpie & k\joule \per \kilogram\\	
		$h_{V} $ & spezifische Schmelzenthalpie & k\joule \per \kilogram\\
		$H $ & Enthalpie & \joule\\		
		$\dot{H}$ & Enthalpiestrom & \joule\per\second\\
		$I$ 	& Stromstärke & A \\
		$m$ & Masse & \kilogram \\
		$\dot{m}$ & Massenstrom & \kilogram\per\second\\
		$n$ & Anzahl & -\\
		$p$ & Druck & \pascal\\
		$p_{D}$ & Dampfdruck & \pascal\\
		$p_{L}$ & Luftdruck & \pascal\\		
		$P$ & Leistung & \watt \\
		$Q$		& Wärmemenge & \joule\\
		$\dot{Q}$ & Wärmestrom & \watt\\
		$R$ & spezifische Gaskonstante & \joule\per(\kilogram\usk\kelvin)\\
		$S$ & Entropie & \joule\per\kelvin\\
		$\dot{S}$ & Entropiestrom & \watt\per\kelvin\\
		$T$ & Temperatur & \kelvin\\
		$t$ & Zeit & \second\\
		%$U$ & innere Energie & \joule\\
		$U$ & elektrische Spannung & V \\
		%$U_{T}$ & Wärmedurchgangskoeffizient & \watt\per(\kilogram\usk\kelvin)\\
		%$h$ & Wärmeübergangskoeffizient & \watt\per(\squaremetre\usk\kelvin)\\		
		$V$ & Volumen & \cubic\meter\\
		$\dot{V}$&Volumenstrom&\cubic\meter\per\second\\
		$w$ & spezifische Leistung & \watt \per \kilogram\\
		$X$	& Wassergehalt & \kilogram \per \kilogram \\
		$Y$ & Wasserbeladung der Luft & \gram\per\kilogram\\
		
\end{longtable}

\begin{longtable}[h]{p{0.15\textwidth} p{0.65\textwidth} p{0.1\textwidth}}
		\caption*{\textbf{griechische Formelzeichen}} \\
		\\
		\textbf{Symbol} & \textbf{Bedeutung} & \textbf{Einheit} \\ %\hline 
		\endhead
		\\
		\multicolumn{3}{c}{Fortsetzung auf der nächsten Seite} \\
		\endfoot
		\multicolumn{3}{c}{ } \\
		\endlastfoot
		
		$\eta_{C}$ & Carnot-Wirkungsgrad & ---\\
		%$\Phi$ & thermische Leistung & \watt\\
		$\varphi$ & relative Feuchte & \% \\
		$\varrho$& Massendichte&\kilogrampercubicmetre\\
		%$\sigma$&Temperaturspreizung&\kelvin\\
		$\vartheta $ & Temperatur  & \degreecelsius\\
		$\Delta\vartheta $ & Temperaturdifferenz  &\kelvin\\
		%$\zeta$ & Druckverlustbeiwert & -\\
		$\psi$ & Sättigungsgrad & \% \\
\end{longtable}

\begin{longtable}[h]{p{0.15\textwidth} p{0.75\textwidth}}
		\caption*{\textbf{Indizes und Abkürzungen}} \\
		\\
		\textbf{Symbol} & \textbf{Bedeutung} \\ %\hline 
		\endhead
		\\
		\multicolumn{2}{c}{Fortsetzung auf der nächsten Seite} \\
		\endfoot
		\multicolumn{2}{c}{ } \\
		\endlastfoot
		
		0 & Referenzzustand (\emph{ambient dead state})\\
		A & Außen/Umgebung\\ 	
		aus & Ausgang\\	
		CV & Kontrollvolumen (\emph{control volume})\\
		%defrost & Abtauvorgang\\
		%DSC & Dynamische Differenzkalorimetrie (\emph{differential scanning calorimetry}) \\
		%e & über die Systemgrenze (\emph{external})\\
		el & elektrisch \\
		EV & Expansionsventil\\
		%F & Volumenstrom\\	
		ein & Eingang \\
		KA & Kälteanlage \\
		KK & Klimakammer\\
		KN & kinetisch\\
		KP & Kompressor\\
		LabVIEW & Programmiersprache und Entwicklungsumgebung für die Messdatenerfassung der Firma National Instruments\\
		L & Luft\\
		m & Mittelwert\\
		Modbus & ein offenes Kommunikationsprotokoll basierend auf einer Master-/Slave-Architektur \\
		MySQL & Datenbank-Verwaltungssoftware \\
		PT & Drucksensor\\
		R & Rücklauf\\
		R 134a & Handelsname von 1,1,1,2-Tetrafluorethan. Eingesetztes Kältemittel\\ 
		rev & Strömungsumkehrung (\emph{reverse})\\
		RTU & entfernte Terminaleinheit (\emph{Remote Terminal-Unit}). Betriebsart für die Datenübertragung für die Modbus-Kommunikation \\
		SPS & Speicherprogrammierbare Steuerung \\
		TT & Temperatur-Sensor\\
		TwinCat & Automatisierungssoftware der Fa. Beckhoff\\
		$\Delta$ t & Zeitschritt der Länge $\Delta$ t\\
		t & technisch\\
		Visual Studio & Integrierte Entwicklungsumgebung für verschiedene Hochsprachen der Fa. Microsoft. Entwicklungsumgebung für TwinCat\\
		VD & Verdampfer \\
		VF & Verflüssiger \\

		
\end{longtable}
\end{onehalfspacing}
