\chapter{Motivation und Ziele}
\label{cha:Motivation_und_Ziele}


Die Motivation zu diesem Forschungsprojekt basiert auf dem Phänomen, dass bei der Unterschreitung des Taupunktes der zu kühlenden vorbei strömenden Luft zunächst Wasser an der Oberfläche des Wärmeübertragers auskondensiert. Unterschreitet die Oberflächentemperatur 0 °C, so können sich erste Kristalle an Keimzellen auf der Oberfläche bilden. Das Wachstum der Kristalle führt zu einer Bereifung des Wärmeübertrager. Die Bereifung führt bei einem konstant bleibendem Volumenstrom durch eine Querschnittsverengung im Wärmeübertrager des Luftkühlers zu einer erhöhten Strömungsgeschwindigkeit. Diese führt zunächst zu einem erhöhten Wärmeübergang und somit zu einer höheren Leistungsübertragung vom Luftkühler auf die Luft.\citep{Schydlo2010}

Eine weitere Bereifung der Lamellen führt zu einem sinkenden Wärmeübergang zwischen der Lamelle und der Luft, da das gefrorene Eis aufgrund der geringen Wärmeleitfähigkeit einen zusätzlichen Wärmeübergangswiderstand darstellt. Um den eintretenden Leistungsabfall zu kompensieren und einen Temperaturanstieg in der Kühlzelle zu vermeiden, wird die Temperatur-Regelung der Kälteanlage, sofern sie dafür ausgelegt ist, die Verdampfungstemperatur verringern. Eine Verringerung der Verdampfungstemperatur führt zu einer Erhöhung der Temperaturdifferenz zwischen Kältemittel und Luft. Eine höhere Temperaturdifferenz steigert wiederum die Kälteleistung. 

Die Verdampfungstemperatur des Kältemittels ist druckabhängig und somit abhängig vom Saugdruck. Um die Verdampfungstemperatur zu verringern, muss der Saugdruck verringert werden. Ein geringerer Saugdruck erhöht die Druckdifferenz zwischen Ein- und Ausgang des Kompressors. Dies führt zu einem erhöhten Stromverbrauch seitens des Kompressors und gleichzeitig sinkt der Wirkungsgrad der gesamten Kälteanlage.
Die Kälteanlage wird im Falle einer Vereisung abgetaut, um danach die ursprüngliche Nennkälteleistung zur Verfügung stellen zu können.

Das Forschungsvorhaben zielt sowohl auf experimentelle als auch numerische Erkenntnisgewinne im Hinblick auf verschiedene Abtaumethoden bei Luftkühlern in einem Kältekreislauf. Diese Erkenntnisse sollen in eine spätere Datenbasis zur energieeffizienten Auslegung von Luftkühlerkomponenten einfließen.

Hierzu wurde bereits eine Kälteanlage geplant und in Betrieb genommen. Der Luftkühler ist hierbei in einer Klimakammer platziert, in der unterschiedlichste Raumbedingungen eingestellt werden können. Die Luftkühler können getauscht werden, um verschiedene Wärmeübertrager-Varianten testen zu können.

Diese Masterarbeit umfasst die Arbeitspakete \textit{Aufbau (bzw. Optimierung) eines Prüfstandes} und \textit{Ermitteln von Messdaten}. Das Arbeitspaket \textit{Aufbau (bzw. Optimierung) eines Prüfstandes} unterteilt sich in 

\begin{itemize}

\item Optimierung des Kältekreislaufes

\item Entwicklung und Bau eines Wägesystems zur Messung und Bestimmung des 2D-Schwerpunktes der Eismenge im Luftkühler

\item Entwicklung und Implementierung einer speicherprogrammierbaren Steuerung (SPS) zur Regelung und Steuerung des Kältekreises und Auswertung des Wägesystems in Echtzeit.

\item Auslegung der Regelparameter für verwendete PID-Regler
\end{itemize}

Nach erfolgreichem Abschließen des ersten Arbeitspaketes wird das zweite Arbeitspaket \textit{Ermitteln von Messdaten} bearbeitet. Dieses setzt sich aus folgenden Punkten zusammen:

\begin{itemize}
\item Durchführung von Reifbildungsversuchen
\item Durchführung von Abtauversuchen mittels verschiedener Abtaumethoden
\item Ermittlung der Reproduzierbarkeit der Messungen.
\end{itemize}
