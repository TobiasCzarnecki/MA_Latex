\chapter{Stand der Technik}
\label{cha:Stand der Technik}

In diesem Kapitel werden alle für die Masterarbeit relevanten theoretischen Grundlagen erläutert. 
Zu erst wird in \ref{sec:Kaeltetechnik} die Thermodynamik der Kältetechnik betrachtet. 

\section{Kältetechnik}
\label{sec:Kaeltetechnik}

Die Kältetechnik wird in verschiedensten Einsatzgebieten eingesetzt, um Kälte zu erzeugen bzw. einem definierten Raum Energie in Form von Wärme zu entziehen. 

Das Konservieren von Lebensmittel ist  der ursprüngliche Hauptzwece der Kältetechnik und ist auch heute noch aktuell. Bereits 3000 Jahre v. Chr. nutzten die Ägypter und Mesopotamier Natureis, um ihre Nahrungsmittel länger haltbar zu machen.\citep{Danfoss2006}

Im Jahre 1834 meldete der US-Amerikaner Jacob Perkins sein Patent zum Thema Kältetechnik in England an. Das Patent beschreibt eine Kaltdampfmaschine in einem geschlossenen Kreislauf mit dem feuergefährlichen Äthyläther als Kältemittel.\citep{Siemens2007}

Carl von Linde baute nach konstruktiven Verbesserungen der Kaltdampfmaschine und Verwendung von Ammoniak als Kältemittel im Jahre 1876 die erste praxistaugliche Kälteanlage. Die ersten Anlagen wurden durch die Maschinenfabrik Augsburg-Nürnberg gebaut und an Brauereien sowie später auch an die Schifffahrt vertrieben.

Mit steigender Bedeutung von Elektrizität als Energieträger nach dem 1.Weltkrieg nahm auch die Entwicklung und der Bedarf an Kälteanlagen zu. Im Jahre 1920 startet die Firma General Electric mit der Serienherstellung von Haushaltkühlschränken mit Hermetik-Verdichtern.

Das vielseitige Gebiet der Kältetechnik umfasst alle Technologien, die zur Bereitstellung von Kälteenergie dienen. Sie unterscheiden sich in der benötigten zuzuführenden Energien, Einsatzbereich und eingesetzten Kältemitteln.  Zu den wichtigsten und heute meist verwendeten Technologien gehören folgende Technologien

\begin{itemize}

\item Kaltdampfkompressionsprozess
\item Absorptionskälteprozess
\item Linde-Verfahren.

\end{itemize}

Weitere nicht so weitverbreitete Technologien der Kältetechnik, jedoch technisch interessante Verfahren  sind zB. das  \textit{Wirbelrohr},  \textit{Magnetische Kühlung} oder das Kühlen mittels einem Peltier-Element. Diese Verfahren werden meist nur unter hohem Energieverbrauch in Sonderfällen angewandt. \citep{Grote2014}

Da sich diese Masterarbeit mit dem Aufbau eines Prüfstandes zur Untersuchung von Abtaumethoden einer Kompressionskälteanlage beschäftigt, wird in den folgenden Kapitel ausschließlich auf diese Technologie eingegangen. Für weitere Informationen bezüglich der anderen Technologien sei an dieser Stelle auf die Literatur \citep{Baehr2013}, \citep{Grote2014} und \citep{Grote2014} verwiesen.


%\subsection{Thermodynamik des Kältekreislauf}
%\label{subsec:Kaeltekreislauf}


%\subsection{•}
%\label{•}

\subsection{log p-h Diagramm}
\label{subsec:log p-h Diagramm}

Bild
Erläuterung
Tabelle von Parametern