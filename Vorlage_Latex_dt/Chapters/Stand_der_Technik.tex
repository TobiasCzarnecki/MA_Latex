\chapter{Stand der Technik}
\label{cha:Stand der Technik}

In diesem Kapitel werden alle für die Masterarbeit relevanten theoretischen Grundlagen erläutert. 
Zu erst wird in \ref{sec:Kaeltetechnik} die Thermodynamik der Kältetechnik betrachtet. 

\section{Kältetechnik}
\label{sec:Kaeltetechnik}

Die Kältetechnik wird in verschiedensten Einsatzgebieten eingesetzt, um Kälte zu erzeugen bzw. einem definierten Raum Energie in Form von Wärme zu entziehen. 

Das Konservieren von Lebensmittel ist  der ursprüngliche Hauptzwece der Kältetechnik und ist auch heute noch aktuell. Bereits 3000 Jahre v. Chr. nutzten die Ägypter und Mesopotamier Natureis, um ihre Nahrungsmittel länger haltbar zu machen.\citep{Danfoss2006}

Im Jahre 1834 meldete der US-Amerikaner Jacob Perkins sein Patent zum Thema Kältetechnik in England an. Das Patent beschreibt eine Kaltdampfmaschine in einem geschlossenen Kreislauf mit dem feuergefährlichen Äthyläther als Kältemittel.\citep{Siemens2007}

Carl von Linde baute nach konstruktiven Verbesserungen der Kaltdampfmaschine und Verwendung von Ammoniak als Kältemittel im Jahre 1876 die erste praxistaugliche Kälteanlage. Die ersten Anlagen wurden durch die Maschinenfabrik Augsburg-Nürnberg gebaut und an Brauereien sowie später auch an die Schifffahrt vertrieben.

Mit steigender Bedeutung von Elektrizität als Energieträger nach dem 1.Weltkrieg nahm auch die Entwicklung und der Bedarf an Kälteanlagen zu. Im Jahre 1920 startet die Firma General Electric mit der Serienherstellung von Haushaltkühlschränken mit Hermetik-Verdichtern.

Das vielseitige Gebiet der Kältetechnik umfasst alle Technologien, die zur Bereitstellung von Kälteenergie dienen. Sie unterscheiden sich in der benötigten zuzuführenden Energien, Einsatzbereich und eingesetzten Kältemitteln.  Zu den wichtigsten und heute meist verwendeten Technologien gehören folgende Technologien

\begin{itemize}

\item Kompressions-Kälteprozess: \textit{Prozess wird angetrieben durch Zufuhr mechanischer Energie}
\item Sorptions-Kälteprozess: \textit{Prozess wird angetrieben durch Zufuhr von Wärmeenergie}
\item Linde-Verfahren.

\end{itemize}

Weitere nicht so weitverbreitete Technologien der Kältetechnik, jedoch technisch interessante Verfahren  sind zB. das  \textit{Wirbelrohr},  \textit{Magnetische Kühlung} oder das Kühlen mittels einem Peltier-Element. Diese Verfahren werden meist nur unter hohem Energieverbrauch in Sonderfällen angewandt. \citep{Grote2014}

Da sich diese Masterarbeit mit dem Aufbau eines Prüfstandes zur Untersuchung von Abtaumethoden einer Kompressionskälteanlage beschäftigt, wird in den folgenden Kapitel ausschließlich auf diese Technologie eingegangen. Für weitere Informationen bezüglich der anderen Technologien sei an dieser Stelle auf die Literatur \citep{Baehr2013}, \citep{Grote2014} und \citep{Grote2014} verwiesen.


\subsection{Kaltdampf-Kälteprozess}
\label{subsec:Kaltdampf-Kälteprozess}

In diesem Abschnitt wird die Thermodynamik des Kaltdampf-Kälteprozesses näher erläutert. Der Kaltdampf-Kälteprozess ist ein linksläufiger \textit{Clausius-Rankine-Kreisprozess}. Die Zustandspunkte des verwendeten Kältemittels im ln p,h Diagramm sind in Abbildung \ref{fig:Schema p-h-Diagramm} dargestellt. 

\begin{figure}[htb]
\centering		
\includegraphics[width=0.75\textwidth]{Pictures/log_p_h_Beahr_Schema.png}
\caption{ln p,h-Diagramm von einem reinen Fluid  \citep{Baehr2013}}
\label{fig:Schema p-h-Diagramm}
\end{figure}

Das halb-logarithmische Diagramm ist ein vielgebrauchtes und hilfreiches Mittel in der Kältetechnik-Branche. 
\footnote{Das Zustandsdiagram wurde vom deutschen Ingenier Richard Mollier (863-1935) im Jahre 1924 erstmalig vorgestellt.} 
Im Diagramm \ref{fig:Schema p-h-Diagramm} ist der Druck logarithmisch auf der y-Achse und die spezifische Enthalpie $h$ auf der x-Achse eingetragen. Die Siedelinie ist durch $x = 0$ und die Taulinie durch $x = 1$. Zwischen $0\leqq x\leqq 1$ befindet sich das Nassdampfgebiet. Es liegt ein Gemisch aus gasförmigen und flüssigem Kältemittel vor. Der Anteil des Gases im Nassdampfgebiet wird durch $x$ ausgedrückt, $1-x$ den Anteil der Flüssigkeit.


In dem Diagramm \ref{fig:Komponeneten und p-h-Diagramm} sind alle Zustandspunkte bei einem Durchlauf des Kältekreislaufes abgebildet.Innerhalb des Nassdampfgebietes verläuft in einem idealen Kältekreislauf, eine Zustandsänderung \textit{isotherm} und \textit{isobar} ab. Wärmezu- oder abfuhr führt nicht zu einer Erhöhung der Temperatur, sondern zu einer Veränderung vom Gas- bzw. Flüssigkeitsanteil. Es wird von einer \textit{latenten}, also nicht fühlbaren,  Wärmeänderung geredet. Um einen Tropfen flüssigen Wasser, dessen Zustand sich auf der Siedelinie befindet, in einen gasförmigigen Zustand, sprich $x=1$ zu überführen, muss ihm die spezifische Verdampfungsenthalpie $\Delta h$ zugeführt werden. Verläuft die Zustandsänderung entgegengesetzt kondensiert der Tropfen und gibt die Verdampfungsenthalpie $\Delta h$ an seine Umgebung ab. 
Außerhalb des Nassdampfgebietes führt eine Wärmezu- oder abfuhr zu einer Veränderung der Temperatur. Die Wärmeänderung ist \textit{sensibel}. 

\begin{figure}[htb]
\centering		
\includegraphics[width=0.6\textwidth]{Pictures/log_p_h_Beahr.png}
\caption{Kreisprozess im ln p,h-Diagramm \citep{Baehr2013}}
\label{fig:Komponeneten und p-h-Diagramm}
\end{figure}


In einem  Kaltdampfprozess, sprich ohne Verluste durch Reibung, finden folgende vier Teilprozesse statt:

\begin{itemize}
\item[1 $\longrightarrow$ 2] Kompression des dampfförmigen Kältemittels unter der Zuführung von elektrischer Leistung $P_{KM}$
\item[2 $\longrightarrow$ 3] Abkühlung, Kondensation und Unterkühlung des Kältemittels unter der Abgabe der Wärmeenergie $\dot{Q}$ über den Verflüssiger an die Umgebung
\item[3 $\longrightarrow$ 4] Entspannung des flüssigen Kältemittels durch das Drosselventil; teilweise setzt die Verdampfung des Fluids ein 
\item[4 $\longrightarrow$ 1] Verdampfung des noch flüssigen Kältemittels auf niedrigem Druckniveau unter der Aufnahme des Wärmestromes $\dot{Q_0}$ aus dem Kühlraum
\end{itemize}

Der Prozess findet auf zwei Durckniveaus statt: dem Verdampfungsdruck $p_V$ und dem Kondensationsdruck $p_K$. Die Verflüssigung des Kältemittels findet auf hohem Druckniveau und die Verdampfung auf niedrigem Druck statt. Die höchste Temperatur wird nach der Kompression am Zustandspunkt 2 erreicht; er befindet sich im Überhitzen- und Hochdruckbereich. Die niedrigste Temperatur ist kurz nach dem Dosselventil und vor dem Verdampfer am Punkt 4. auf niedrigem Druckniveau. 

Nach dem Anwenden des 1. Hauptsatzes der Thermodynamik, die Erhaltung der Energie in einem System, auf den Kältekreislauf folgt die Gleichung :

 \begin{equation}
 	|\dot{Q}|  = \dot{Q_0} +  P_{KM}.
 	\label{eq:Energiebilanz}
 \end{equation}
 
Die elektrische Antriebsleistung der Kältemaschine ist die aufgenommene elektrische Leistung durch den Kompressor zwischen den Zustandspunkten 1  und 2. Sie ergibt sich zu:

\begin{equation}
P_{KM} = \dot{m}~ w_t= \dot{m}~ (h_2 - h_1) = \frac{\dot{m} }{\eta_{sV}} (h_{2'}- h_1).
\label{eq:Antriebsleistung}
\end{equation}

Hierbei ist $\eta_{sV}$ der isentrope Wirkungsgrad des Kompressors. Der isentrope Wirkungsgrad setzt den realen Kältekreislauf in ein Verhältnis zum idealen Kältekreislauf. Die Überhitzung des Gases am Austritt des Kompressors ist höher als die Überhitzung nach einer isentropen Verdichtung. Daraus folgt eine höhere Energieaufnahme durch den Kompressor und ein höherer Wärmestrom $\dot{Q}$, der über den Verflüssiger an die Umgebung abgegeben werden muss. Der isentrope Wirkungsgrad ist definiert über 

\begin{equation}
\eta_{sV}:= \frac{h_{2'}- h_{1}}{h_2 - h_1}.
\label{eq:Antriebsleistung}
\end{equation}


Der Wärmestrom $\dot{Q}$ wird über den Verflüssiger zwischen den Zuständen 2 und 3 abgeführt. Die Formel von  $\dot{Q}$ lautet: 

\begin{equation}
	\dot{Q} = \dot{m}~q_0 = \dot{m}~ (h_3 - h_2)< 0.
	\label{eq:Wärmestrom}
\end{equation}

Der Wärmestrom $\dot{Q}$ ist immer kleiner als Null; er wird dem Kreislauf folglich entzogen.  
 
Über ein Drosselorgan wird das Kältemittel vom hohen Druckniveau auf das niedrigere Druckniveau entspannt. Der Teilprozess findet zwischen den Zustandspunkten 3 und 4 statt und wird als $isenthalp$ angenommen.  
 
Die Kälteleistung $\dot{Q_0}$, sprich den aus dem Kühlraum zu entnehmender Wärmestrom, ergibt sich aus dem Kältemittel-Massenstrom $\dot{m}$ und den spezifischen Enthalpien der Zustände 4 und 1 :

\begin{equation}
	\dot{Q_0} = \dot{m}~ q_0 = \dot{m}~ (h_1 - h_4).
	\label{eq:Kälteleistung}
\end{equation}




Die Bewertung einer Kälteanlage erfolgt durch die Leistungszahl $\epsilon_{KM}$: 

\begin{equation}
	\epsilon_{KM} := \frac{Kälteleistung}{Antriebsleistung} =\frac{\dot{Q_0}}{P_{KM}}.
	\label{eq:Leistungszahl}
\end{equation}

Ziel bei der Auslegung und dem Betrieb einer Kältemaschine ist eine möglichst große Leistungszahl zu erlangen. Damit $\epsilon_{KM}$ groß wird, muss die Kälteleistung $\dot{Q_0}$ groß werden und die aufgewandte Verdichterleistung $P_{KM}$ klein werden. 


\subsection{Komponenten eines Kaltdampfprozesses}
\label{subsec:Komponenten eines Kaltdampfprozesses}

Die Komponenten für einen einfachen Kaltdampfprozess  besteht aus vier Komponenten:

\begin{itemize}
\item der Kompressor
\item der Verflüssiger 
\item das Drosselvenil Expansionsventil
\item der Verdampfer. 
\end{itemize}

In Abbildung \ref{fig:einfacher Kältekreislauf} sind die vier Komponenten mit ihren Zustandspunkten dargestellt.

\begin{figure}[htb]
\centering		\includegraphics[width=0.50\textwidth]{Pictures/Kaltekreislauf_beahr.png}
\caption{Einfacher Kältekreislauf \citep{Baehr2013}}
\label{fig:einfacher Kältekreislauf}
\end{figure}

\subsubsection*{Der Kompressor}
Der Kompressor bildet das Herzstück der Kälteanlage. Er verdichtet das gasförmige Kältemittel von niedrigem Druck auf ein höheres Druckniveau. Um diese Arbeit zu verrichten, wird der Verdichter mit elektrischer Energie versorgt. Der Kompressor gibt es in verschiedenen Bauvarianten. Die zwei wichtigsten Bauvarianten sind der \textit{Hubkolbenverdichter} und der \textit{Rotationskolbenverdichter}. Die Baugruppen der Verdichter werden in offene, halbhermetische und vollhermetische Verdichter unterschieden. Schrauben-,Scroll- sowie Turboverdichter sind Bauarten der \textit{Rotationskolbenverdichter}. 



Ein wichtiges Kriterium bei Verdichtern ist das Druckverhältnis von Ansaugdruck, vor der Kompression, und dem Ausgangsdruck. Das Druckverhältnis $\pi$ ist definiert als:



\begin{equation}
\pi := \frac{p_{aus}}{p_{ein}}.
\label{Druckverhältnis}
\end{equation}

\subsubsection*{Der Verflüssiger}

Dem Kältemittel wird im Verflüssiger auf einem hohen Druckniveau Wärme entzogen. Der Verflüssiger kühlt das überhitzte, gasförmige Kältemittel ab. Beim Austritt aus dem Verflüssiger ist das Kältemittel meist vollständig kondensiert. 
Um einen Wärmeentzug zu bewerkstelligen gibt es drei Bautypen:

\begin{itemize}
\item Wassergekühlte Verflüssiger
\item Luftgekühlte Verflüssiger
\item Verdunstungsverflüsssiger
\end{itemize}

Wassergekühlte Verflüssiger können, aufgrund der besseren Wäärmeübertragung verglichen zu Luft, sehr kompakt gebaut werden. Eine typische Bauform ist das \textit{Bündelrohrverflüssiger}.
In der Praxis werden am häufigsten luftgekühlte Verflüssiger eingesetzt. Um die gleiche Kühlleistung wie ein wassergekühlter Verflüssiger zu erreichen, werden Lamellen und Ventilatoren eingesetzt. Die Lamellen vergrößern die Fläche für die Wärmeübertragung mit der Luft. Ventilatoren ermöglichen durch einen höheren Luftdurchsatz und der daraus resultierendem höhere Wärmeübertragung eine größere Kühlleistung und eine kompaktere Bauform der Wärmeübertragers. Diese Variante hat den Vorteil, dass die einen wartungsfreien Betrieb  sowie eine einfache Reinigung ermöglicht.


\subsubsection*{Das Expansionsventil}

Das Expansionsventil versorgt den Verdampfer mit dem nötigen Kältemittel-Massenstrom. Die Zuführung des Kältemittels erfolgt über eine Druckdifferenz. Durch eine lokale Verengung des Strömungquerschnitts, verringet sich der Druck des durchfließenden Kältemittels. Das Kältemittel vergrößert sein Volumen und es kommt zur Expansion. Die Druckreduzierung erfolgt ohne zusätzliche Arbeit. Im idealen Fall wird bei diesem Prozess auch keine Wärme abgefuhrt; der Prozess ist \textit{isenthalb}. 
Das Expansionsventil trennt zusammen mit dem Kompressor die zwei Druckseiten des Kältekreislaufes. Es gibt Expansionsventile sowohl als regelbare und nicht regelbare Ausführungen. Bei kleineren Anlagen erfolgt die Expansion ungeregelt zum Beispiel durch Kapillarrohre. Geregelte Expansionsventile werden in mittleren und großen Kälteanlagen eingesetzt. Die Regelung erfolgt durch die Querschnittsänderung und dem damit einhergehendem Druckabfall.  

\subsubsection*{Der Verdampfer}

In dem Verdampfer wird das Kältemittel eingespritzt. Das Kältemittel verdampft und entzieht seiner Umgebung dabei Wärme. Aufgrund der vielfältigen Anforderungen an Verdampfer, gibt es eine Vielzahl an Bauarten für Verdampfer. Mögliche Bauarten sind 

\begin{itemize}
\item Glattrohrverdampfer
\item Beripptes Verdampferregister
\item Rippenrohrverdampfer
\end{itemize}

Um eine möglichst große Kälteleistung zu ermöglichen, werden wie beim Verflüssiger auch, Ventilatoren eingesetzt. Die Ventilatoren erzwingen einen Luftstrom durch den Verdampfer und erhöhen damit die Wärmeübertragung zwischen der Luft und den Verdampferrohren. 



\section{Thermodynamik der feuchten Luft}
\label{sec:Thermodynamik der feuchten Luft}

Ein Verdampfer hat die Aufgabe einer Umgebung Energie in Form von Wärme zu entziehen. Hierfür wird in einem Wärmeübetrager flüssiges Kältemittel verdampft. Das verdampfende Kältemittel kühlt zunächst den Wärmeübertrager, danach wird über die Wärmeübertrager-Lamellen der vorbeiströmende Luft Wärme entzogen. 

Bei dem Kühlprozess durch einen Verdampfer kommt es zwischen dem Wärmeübertrager und der feuchten Luft verschiedenen thermodynamischen Phänomenen. Die auftretenden Phänomene lassen sich in folgende Gruppen einordnen:

\begin{itemize}
\item	Abkühlung der feuchter Luft 
\item 	Wassertropfenbildung auf Lamellenoberfläche 
\item	Kristallbildung auf Lamellenoberfläche.
\end{itemize}

In den nachfolgenden Abschnitten \ref{subsec:Feuchte Luft} und \ref{subsec:Reifbildung} werden die Phänomene kurz erklärt.

\subsection{Feuchte Luft}
\label{subsec:Feuchte Luft}

\begin{figure}[htb]
\centering		\includegraphics[width=0.85\textwidth]{Pictures/h_x_Diagramm_Beahr.png}
\caption{$h^*$, $X$- Diagramm für feuchte Luft bei einem Gesamtdruck $p= $ 100 kPa \citep{Baehr2013}}
\label{fig:h_x_diagramm}
\end{figure}


Bei dem Wärmeentzug durch den Verdampfer wird  zunächst die noch nicht gesättigte feuchte Luft abgekühlt. Luft besitzt die Eigenschaft eine bestimmte Menge Wasser aufnehmen zu können. Warme Luft kann mehr Wasser aufnehmen als kalte Luft. Das Verhältnis von der aufgenommene Masse Wasser zur Masse Luft ist definiert als Beladung: 

\begin{equation}
X = \frac{m_W}{m_L}.
\label{eq:Beladung}
\end{equation}

In dieser Formel bezieht sich $m_W$ auf gasförmige, flüssige oder feste Form von Wasser und $m_L$ auf die Masse der trockenen Luft.  $X$ kann werden zwischen 0 für trockene Luft und $\infty$ für reines Wasser annehmen. In der Regel bleibt $X$ jedoch kleiner als 0,1. 

Die absolute Feuchte ist definiert als das Verhältnis von Masse des Wasserdampfers $m_W$ zum eingenommen Volumen $V$ der feuchten Luft. Die Formel lautet: 

\begin{equation}
\varrho := \frac{m_W}{V} = \frac{p_w}{R_W T}.
\label{eq:Absolute Feuchte}
\end{equation}


Die relativen Feuchte $\varphi$ ist das Verhältnis der absoluten Feuchte im Verhälnis zur Maximalwert oder Sättigungswert der absoluten Feuchte: 

\begin{equation}
 \varphi := \frac{\varrho}{\varrho_W^s}.
 \label{eq:Rel Feuchte}
\end{equation}

Wird nun in die Gleichung \ref{eq:Beladung} die relative Feuchte aus Gleichung \ref{eq:Rel Feuchte} eingesetzt, ergibt sich eine weitere Formel für die Beladung $X$. 

\begin{equation}
X = \frac{m_W}{m_L} = 0,622 \frac{p_{W}^s}{p/\varphi - p_{W}^s}.
\label{eq:Beladung 2}
\end{equation}

Hierbei ist $p_{W}^s$ der Partialdruck des Sättigungspunktes.
Betrachtet man nur die Wasserdampfbeladung der Luft, so lässt sich feststellen, dass die maximale Menge an aufzunehmenden Wasser einen Grenzwert hat. Dieser Grenzwert wird Sättigungswert der Wasserdampfbeladung genannt und ist eine Funktion abhängig von der Temperatur und dem Druck.  Sie berechnet sich nach dem Gesetz von Dalton zu :

\begin{equation}
 X_s (T,p) = 0,622 \frac{p_{W}^s}{p - p_{W}^s}.
 \label{eq:Sättigungsbeladung}
\end{equation}

Bei der Abkühlung von feuchter Luft kann der Fall eintreten, dass $X > X_s$ ist. Sprich die Beladung der trockenen Luft ist größer als das alles im Wasserdampf aufgenommen werden kann. Der erste Tropfen Kondensat bildet sich und Wasser fällt aus. Es ergibt sich eine Kondesationsmenge von:

\begin{equation}
\Delta X m_L = (X - X_s)m_L.
\label{eq:Delta_X}
\end{equation}

Die Wassermenge $X_s m_L$ ist gasförmig von der trockenen Luft gespeichert. Die Kondensationsmenge $\Delta X m_L$ setzt sich nun an Keimpunkten in der Umgebung ab oder wird als Nebel aus der Gasphase ausgeschieden. Keimzellen für Wassertropfen können zum Beispiel Verunreinigungen oder raue Oberflächen sein. Das höchste treibende Potential für eine solche Keimzelle hat der kälteste Punkt im System. In unserem Anwendungsfall ist das der Wärmeübertrager des Verdampfer. Da der Verdampfer zur Bereitstellung der Kälteleistung und einer funktionierenden Wärmeübertragung stets eine Temperaturdifferenz zwischen dem Kältemittel und der vorbei strömenden Luft  bereitstellt,  bilden sich hier die ersten Tropfen. Je nach ursprünglicher Beladung der Luft bilden sich die Tropfen eher am Anfang oder Ende des Wärmeübertrager. War die feuchte Luft schon vor dem Eintritt in den Wärmeübertrager bereits stark gesättigt, bilden sich Tropfen am Eingang der Lamellen. 
\citep{Baehr2013}


\subsection{Reif- und Eisbildung}
\label{subsec:Reifbildung}

Liegt die Oberflächentemperatur auf dem Wärmeübertrager des Verdampfers nicht nur unter dem Taupunktpunkt der feuchten Luft sondern auch unter dem Gefrierpunkt, kann es zum Gefrieren der kondensierten Tropfen und/oder zur Desublimation von Wasserpartikel auf der Oberfläche kommen. Dieser Abschnitt soll einen Überblick über diese zwei thermodynamischen Phänome geben. Da in dieser Arbeit nicht der Reifbildungsprozess im Hauptfokus steht, sondern  der technische Aspekt der Abtauung, wird der Eisbildungsprozess hier nur kurz erläutert.

In der Literatur gibt es zahlreiche Quellen, die sich mit der Reif- bzw. Eisbildung auseinander setzen. Die Quellen beschreiben den Kristallbildungsprozess sowohl aus der rein theoretisch Sicht als auch simulationsrelevanten und technischen Überlegungen und Untersuchungen. Der scheinbar triviale Prozess von der Bildung eines Eiskristalles auf einer Oberfläche und sein weiteres Wachstumsverhalten ist sehr komplex und Gegenstand zahlreicher aktueller und schon abgeschlossener Forschungsprojekten.

Neben den theoretischen Grundlagen wird in der Arbeit von \textsc{\citeauthor{Schydlo2010}} ein Simulationsmodell für den Reifbildungs- und Abtauprozess auf einem Rohr entwickelt. Zudem sind bisherigen Arbeiten zu der Thematik in \citep{Schydlo2010} aufgelistet und zusammengefasst. 
Praktische Untersuchungen sowie Versuchsaufbauten zum Thema der Vereisung von Luftkühler werden in \textsc{\citeauthor{Sahinagic2004}} und \textsc{\citeauthor{Kosowski2009}} beschrieben. In den Arbeiten wurden die Vereisungs- und innovative Abtauungsprozesse von einer CO$_2$-Wärmepumpe, die zur Heizung von Passivhäuser eingesetzt wird, untersucht.   



Es gibt zahlreiche Einflussgrößen die auf den Prozess und die Form des Eiskristalls und späteren Reif einwirkten. 
  Die wichtigsten Einflussgrößen sind die Luftgeschwindigkeit, die Lufttemperatur, die Luftfeuchte, die Oberflächentemperatur und die Zeit. Um eine den Reif charakterisieren zu können, werden folgende Größen zur Hilfe genommen die Reifdicke, die Reifdichte, die Porösität und die Wärmeleitfähigkeit. 

In der Arbeit von \textsc{\citeauthor{Hayashi1977}} aus dem Jahre 1977 wird der Eiskristallwachstum in drei Phasen unterteilt:

\begin{enumerate}
\item Eindimensionales Kristallwachstum
\item Reifschichtwachstumsphase 
\item Vergletscherung.
\end{enumerate}


\begin{figure}[htb]
\centering		\includegraphics[width=0.85\textwidth]{Pictures/Reifbildungsphasen_Schydlo.png}
\caption{Kristallwachstum auf einer ebenen Oberfläche \citep{Schydlo2010}}
\label{fig:Kristallwachstum}
\end{figure}


In Abbildung \ref{fig:Kristallwachstum} sind die drei Kristallwachstums-Phasen nach \textsc{\citeauthor{Hayashi1977}} sowie die vorhergehende Keimbildungsphase, , eingeführt in \citep{Sahinagic2004}, dargestellt. 

\subsection*{Keimbildungsphase}

In der Keimbildungsphase bilden sich zunächst Wassertropfen auf der Lamellenoberfläche, die trotz Temperaturen kleiner als der Gefrierpunkt nicht erstarren, sondern wachsen zu größeren Tropfen an. Je kleiner die Unterkühlung desto größer werden die Tropfen bevor sie erstarren und in die erste Kristallwachstumsphase übergehen. 

\subsection*{Eindimensionales Kristallwachstum}
Die erste Phase ist gekennzeichnet durch Kristallwachstum senkrecht zur Oberfläche und mit einheitlicher Wachstumsgeschwindigkeit. Dies führt zu einer erhöhten Rauigkeit aufgrund der sich stetig vermehrenden Kristalle. 

\subsection*{Reifschichtwachstumsphase}
In der zweiten Phase beginnt das dreidimensionale Wachstum. Die Kristalle fangen an sich miteinander zu verästeln. Ein poröses Kristallgitter entsteht. Aufgrund des  Wärmeleitwiderstandes, der mit der Reifdicke steigt, erhöht sich die Oberflächentemperatur der Reifschicht. Desweiteren kommt es zu einem Massenstrom innerhalb der Reifschicht, ausgelöst durch Diffusion. Die Diffusion rührt aus  den  Konzentrationsunterschiede zwischen der Lamelle und der Reifoberfläche. Der Wassermassenstrom läuft in das poröse Kristallgitter und gefriert dort in Nähe der Lamelle. Die Dichte der Reifschicht steigt und mit ihr der Wärmewiderstand. Dies führt zu einer Erhöhung der Oberflächentemperatur der Reifschicht und schließlich zur Überschreitung des Gefrierpunktes von Eis. Die Spitzen des Kristalle schmelzen und es bildet sich Kondensat. Die dritte Wachstumsphase beginnt. 

\subsection*{Vergletscherung}

Das flüssige Wasser läuft aufgrund der Kapillarwirkung der Kristalle in die Zwischenräume des Kristalgitters und gefriert dort wieder. Die Kristallgitter werden dichter und kompakter. Dies führt zur Steigerung der Wärmeleitfähigkeit der Reifschicht. Es wird von einer Vergletscherung gesprochen. Die Oberflächentemperatur der Reifschicht sinkt erneut und fällt erneut unter den Gefrierpunkt. Nun kann die feuchte Luft erneut an der Oberfläche des Reifs desublimieren. Der Prozess wiederholt sich solange bis das Eis so kompakt ist, dass kein weiteres Kondensat mehr in die Reifschicht eindringen kann. 
    





\subsection{Morphologie von Eiskristallen}
\label{subsec:Morphologie}





